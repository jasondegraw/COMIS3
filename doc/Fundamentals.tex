\documentclass[12pt]{report}
\usepackage{titlesec}
\usepackage{cite}

%\setlength{\parindent}{0cm}
%\setlength{\parskip}{1em}

%\setcounter{tocdepth}{3}
%\setcounter{secnumdepth}{3}

%\titleclass{\subsubsubsection}{straight}[\subsection]
%
%\newcounter{subsubsubsection}[subsubsection]
%\renewcommand\thesubsubsubsection{\thesubsubsection.\arabic{subsubsubsection}}
%\renewcommand\theparagraph{\thesubsubsubsection.\arabic{paragraph}} % optional; useful if paragraphs are to be numbered
%
%\titleformat{\subsubsubsection}
%  {\normalfont\normalsize\bfseries}{\thesubsubsubsection}{1em}{}
%\titlespacing*{\subsubsubsection}
%{0pt}{3.25ex plus 1ex minus .2ex}{1.5ex plus .2ex}
%
\makeatletter
\newcommand\subsubparagraph{\@startsection{paragraph}{5}{\z@}%
                                      %{3.25ex\@plus 1ex\@minus .2ex}
                                      %{1.5ex \@plus .2ex}
                                      {-3.25ex\@plus -1ex \@minus -.2ex}%
                                      {1.5ex \@plus .2ex}
                                      {\normalfont\normalsize\bfseries}}
\renewcommand\paragraph{\@startsection{paragraph}{4}{\z@}%
                                      {-3.25ex\@plus -1ex \@minus -.2ex}%
                                      {1.5ex \@plus .2ex}
                                      {\normalfont\normalsize\bfseries}}
\renewcommand\subparagraph{\@startsection{subparagraph}{5}{\z@}%
                                         {-3.25ex\@plus -1ex \@minus -.2ex}%
                                         {1.5ex \@plus .2ex}
                                         {\normalfont\normalsize\bfseries}}
%%\renewcommand\paragraph{\@startsection{paragraph}{5}{\z@}%
%%  {3.25ex \@plus1ex \@minus.2ex}%
%%  {-1em}%
%%  {\normalfont\normalsize\bfseries}}
%\renewcommand\subparagraph{\@startsection{subparagraph}{6}{\parindent}%
%  {3.25ex \@plus1ex \@minus .2ex}%
%  {-1em}%
%  {\normalfont\normalsize\bfseries}}
%\def\toclevel@subsubsubsection{4}
%\def\toclevel@paragraph{5}
%\def\toclevel@paragraph{6}
%\def\l@subsubsubsection{\@dottedtocline{4}{7em}{4em}}
%\def\l@paragraph{\@dottedtocline{5}{10em}{5em}}
%\def\l@subparagraph{\@dottedtocline{6}{14em}{6em}}
\makeatother

\setcounter{secnumdepth}{6}
\setcounter{tocdepth}{6}

\begin{document}
\begin{titlepage}
\begin{flushright}
LBL-28560
\end{flushright}
\vspace{0.75in}
\begin{center}
\Huge
\textbf{COMIS Fundamentals}
\vspace{1in}

\Large
Edited by:

Helmut E. Feustel and Alison Rayner-Hooson

\vspace{0.75in}

Applied Science Division\\
Lawrence Berkeley Laboratory\\
Berkeley, CA 94720

\vspace{0.5in}

May 1990
\end{center}
\end{titlepage}
\begin{center}
\textbf{DISCLAIMER}
\end{center}

\noindent~This report was prepared as an account of work sponsored by an
agency of the United States Government. Neither the United States
Government nor any agency Thereof, nor any of their employees,
makes any warranty, express or implied, or assumes any legal
liability or responsibility for the accuracy, completeness, or
usefulness of any information, apparatus, product, or process
disclosed, or represents that its use would not infringe privately
owned rights. Reference herein to any specific commercial product,
process, or service by trade name, trademark, manufacturer, or
otherwise does not necessarily constitute or imply its endorsement,
recommendation, or favoring by the United States Government or any
agency thereof. The views and opinions of authors expressed herein
do not necessarily state or reflect those of the United States
Government or any agency thereof.
\newpage
\begin{flushleft}
{\Huge
\textbf{COMIS - \textit{Fundamentals}}
}

\vspace{0.25in}
\textbf{Edited by: Helmut E. Feustel and Alison Rayner-Hooson}
\vspace{0.25in}

{\Large
\textbf{Francis Allard}\\
France

\textbf{Viktor B. Dorer}\\
Switzerland

\textbf{Helmut E. Feustel}\\
USA

Eduardo Rodriguez Garcia\\
Spain

Mario Grosso\\
Italy

Magnus K. Herrlin\\
Sweden

Liu Mingsheng\\
Peoples Republic of China

Hans C. Phaff\\
Netherlands

Yasuo Utsumi\\
Japan

Hiroshi Yoshino\\
Japan
}

\vspace{0.25in}
 For current addresses please check the Appendix
 \end{flushleft}

\newpage
\noindent{\Large\textbf{Foreword}}\\
\noindent
The COMIS workshop (Conjunction of Multizone Infiltration Specialists) was a
joint research effort to develop a multizone infiltration model. This workshop
(October 1988 - September 1989) was hosted by the Energy Performance of
Buildings Group at Lawrence Berkeley Laboratory's Applied Science Division.
The task of the workshop was to develop a detailed multizone infiltration program
taking crack flow, HVAC-systems, single-sided ventilation and transport mechanism through large openings into account. This work was accomplished not by
investigating into numerical description of physical phenomena but by reviewing
the literature for the best suitable algorithm. The numerical description of physical phenomena is clearly a task of EA-Annex XX ``Air Flow Patterns in Buildings'', which will be finished in September 1991. Multigas tracer measurements
and wind tunnel data will be used to check the model. The agenda integrated all
participants' contributions into a single model containing a large library of
modules. The user-friendly program is aimed at researchers and building professionals.
From its announcement in December 1986, COMIS was well received by the
research community. Due to the internationality of the group, several national
and international research programmes were co-ordinated with the COMIS
workshop. Colleagues from France, Italy, Japan, The Netherlands, People's
Republic of China, Spain, Sweden, Switzerland, and the United States of America
were working together on the development of the model.
Even though this kind of co-operation is well known in other fields of research,
e.g., high energy physics; for the field of building physics it is a new approach.
The COMIS Fundamentas contains an overview about infiltration modelling as
well as the physics and the mathematics behind the COMIS model.

\begin{flushleft}
Helmut E. Feustel\\
COMIS Co-ordinator\\
Berkeley, California\\
January 31,1990\\
\end{flushleft}

\newpage
\begin{center}
%{\Large\textbf{CONTENTS}}
\end{center}
\tableofcontents

Nomenclature
\chapter{Infiltration Modelling}
\section{Introduction}
\section{History of Infiltration Modelling}
\section{Multizone Infiltration Models}
\section{Review}
\section{Conclusion}
\chapter{Physical Fundamentals}
\section{Pressure Distnbutlon}
\subsection{Introduction}
\subsection{Wind Pressure}
\subsubsection{Modelling Wind Pressure Distribution}
\paragraph{Reference Data}
\paragraph{Methodology}
\paragraph{Analysis}
\paragraph{Calculation Model}
\paragraph{Outlook}
\subsection{Thermal Buoyancy}
\section{Air Flows Through Openings}
\subsection{Air Flow through Cracks}
\subsubsection{Introduction}
\subsubsection{Flow Equation}
\paragraph{Duct Flow}
\paragraph{Crack Flow}
\paragraph{Temperature Influence on Crack Flow}
\paragraph{Equation Form Influence}
\paragraph{Regression}
\paragraph{Temperature Equation}
\subparagraph{Classification of Crack Forms}
\subparagraph{Door or Single-Pane Window}
\subparagraph{Prime-and-Storm Window}
\subparagraph{Walls}
\subsubparagraph{Tiny Cracks}
\subsubparagraph{Wide Cracks}
\subsubsection{Conclusion}
\subsection{Air Flow through Large Openings}
\subsubsection{Introduction}
Airflow through doorways, windows and other large openings are a significant
way in which air, pollutants and thermal energy are transferred from one zone of
a building to another or to the outside. In a previous review of multizone infiltration models made in 1985, Feustel and Kendon \cite{feustel_kendon_1985} pointed out that no code was
able to solve this problem other than dividing the large opening into a series of
small ones described by crack flow equations.

The subject of air flows through large openings in fact includes a large number of
different problems ranging from steady gravitational flows to fluctuating flows
due to wind turbulence, and including recirculating flows caused by boundary
layer effects in a thermally driven cavity.

COMIS’ contribution has been to describe the various physical problems, review
the solutions already developed in the literature and put forward a general proposal dealing with most of the large openings included in the pressure network
modeling of a multizone building \cite{allard_utsumi_1992}.

\subsubsection{Short Review of Literature}
Air may flow differently at the top of a doorway than at the bottom. This bidirectional flow, characteristic of large opening behavior, has a variety of causes
which can be classified broadly in two categories - those which induce steady
flow by virtue of their mean value and those whose effect is due to their fluctuating nature. 

\paragraph{Steady Flows through Large Openings}
The first category includes the effects of mean wind velocity, the gravitational
flows due to density gradients, the effects of boundary layer flows developed in a
cavity and the coupling effects induced by ventilation or heating systems.

Many authors have dealt with gravitational flow approach through vertical openings
\cite{shaw_whyte_1974}, \cite{brown_solvasson_1962, shaw_1976, lidwell_1977} and provided
elementary solutions based on purely natural or
mixed convection. Nevertheless, most of them refer to the basic solution
developed analytically by Brown and Solvasson \cite{brown_solvasson_1962}. This solution, based on
Bernoulli’s assumption, delivers the mass flow-through the opening as a function
of the main characteristics of air on both sides of it.

Some improvement on this approach has been made by taking into account temperature stratification on each side of the opening.

In the various experimental studies developed by Balcomb et al. \cite{balcomb_yamaguchi_1983} Weber \cite{weber_1980}
Hill et al. \cite{hill_et_al_1986} and Pelletret \cite{pelletret_khodr_1989} it appears that
most of the steady state configurations can be well represented by uniform density gradients. 

Finally, among the various problems found in defining large opening behavior
horizontal openings should be mentioned. The most interesting aspect of horizontal openings is when the fluid above the opening has a greater density than the
fluid below. In this case an unstable condition is reached and an exchange occurs
between the lighter fluid flowing upward and the heavier fluid flowing downward. This problem was first studied
by Brown \cite{brown_1962} but few authors have made a consistent contribution towards solving it.

In a recent paper Epstein \cite{epstein_1988} discusses the buoyancy exchange flow through
small openings in horizontal partitions. He presents an experimental study using
brine above the partition and fresh water below it and identifies four different
regimes as the aspect ratio of the opening is increased. This paper gives a good
description of the flow processes but it is limited to purely natural convection
effects so that the results are difficult to extend to more general configurations.

Much more work is needed to describe accurately the behavior of horizontal
openings in natural or mixed convection configurations. The few papers found in
the literature are not sufficient to do so and they do not allow us to include such
configurations in a multizone infiltration code.
 
\paragraph{Unsteady Flows through Large Openings}
The second category usually represents two very different kinds of configurations:
\begin{itemize}
\item The transient behavior of flows due to slow evolution of the boundary conditions
\item The fluctuating air flows due to fluctuating pressures or velocities
\end{itemize}

\subparagraph{Transient Flows}
The first type of problem is usually solved with a steady state description of the
dynamical process. Bernouilli’s flow theory or any other correlation is used to
calculate the steady state flow corresponding to the boundary conditions of each
time step. The main assumption in this case is that the flow will be fully
developed instantaneously and will follow the imposition of new boundary conditions at each time step.

The variation of the boundary conditions can be given by hourly variation of
average wind speed and direction, by a schedule of occupancy, or the working of
mechanical ventilation systems, or by the evolution in time of the thermal state of
the building itself.

The most common way of solving the problem is to combine the air flow model
with other models dealing with thermal simulation or any other phenomena driving the air flow process and to give a complete description of the coupled problems. In some cases, however, it is better to use a simpler approach and to give
an estimate of the evolution in time of the flow through an opening caused by a 
single perturbation. In this way Kiel and Wilson \cite{kiel_wilson_1986} studied the unsteady buoyancy flow that occurs during the door opening or closing period. They investigated the decrease in flow rate that results from the effect of limited interior
volume and finally considered the effect of door pumping.

Van der Maas et al. \cite{vandermaas_et_al_1989} looked at the same problem from a different point of
view. They studied the evolution of the air flow rate and temperature in a room
when opening a window. Assuming insulated walls and developing the energy
balance of the inside air, they show by full-scale experiments that a reasonable
agreement is obtained. The main advantage of such a solution is the avoidance of
coupling with a real simulation of the thermal behavior of the whole building and
the ability to give a quick estimate of the transient evolution of the air flow

\subparagraph{Fluctuating Flows through Large Openings}
Experimental results have shown that the fluctuating effects can be particularly
significant in the case of one sided ventilation or when the wind direction is parallel to openings in two parallel facades.
Nevertheless, as pointed out by Vandaele and Wouters \cite{vandaele_wouters_1989} in their
review paper very few correlations have been proposed
and most of them concern very particular configurations.

The fluctuating flows are mainly created by the effects of turbulence due to local
wind characteristics. They depend on the turbulence characteristics of the incoming wind
 and on the turbulence induced by the building itself. These mechanisms
have been studied on scale models or full scale cells and wind tunnel visualizations have
 been used to describe the main phenomena. Turbulences in the air
flow along an opening cause simultaneous positive and negative pressure fluctuations of the inside air.
The studies of Cockroft et al. \cite{cockroft_robertson_1976} Warren \cite{warren_1977} Crommelin
and Vrins \cite{crommelin_vrins_1988} and Etheridge \cite{etheridge_1984} illustrate this problem.

Because of the complexity of the problem and the high number of parameters
influencing the development of the eddies along the facade of a building very few
correlations have been proposed and implemented in multizone air flow models.

Phaff and de Gids \cite{phaff_degids_1980} propose an empirical correlation deduced from experimental
work and give a general definition of the ventilation rate through an open window as a function of
temperature difference, wind velocity and fluctuating terms.
Since they found an existing flow without either wind or temperature difference a
fixed turbulent term was added to the volumetric flow. This turbulent term is
presented as an additional pressure drop written as a function of the flow
exponent.

These types of models, combining the effects of wind velocity, temperature
difference and turbulence, should be able to predict the effect of fluctuating wind
on the air flow exchange through an external opening in a multizone pressure network.
However, much work is needed in order to give a precise description of the
effects of the many parameters influencing the structure of the local wind at the
very location of the opening. 

\subsubsection{Integration into COMIS}
A multizone infiltration model like COMIS is defined by a network description of
the pressure field in a building. The pressure nodes represent control volumes or
zones at thermodynamical equilibrium. They are linked together by nonlinear
flow equations expressed in terms of pressure difference. Mass conservation of air
in each zone leads to a nonlinear system of pressure equations.

These are two ways of describing the behavior of large openings. Either we can
describe the air flow rate through a large opening using a nonlinear equation of
the pressure drop, or we can solve the problem separately and include the result as
an unbalanced flow in the mass conservation equation of the described zone. 

\subsubparagraph{Existing Solutions}
To represent the behavior of a large opening in an air flow model we can take an
explicit definition of the flow given by a Bernoulli flow regime assumption, or
any other correlation of natural or mixed convection \cite{barakat_1987} corresponding to the
configuration. In the case of both methods the large opening is disconnected from
the general pressure network, is solved separately and then represented by an
unbalanced flow in the mass conservation equation of each zone. Since the flow
is strongly dependent on the pressure field around and inside the building the first
solution will lead to a high number of iterations and a direct approach appears
more attractive.

To connect the large openings to the general pressure network it is necessary to
define their behavior in terms of nonlinear equations of the pressure drop. The
first possibility is to describe the large opening as a conjunction of parallel small
openings, properly located, and with only a one-way flow allowed for each one.
Each small opening is then described by a crack flow equation taking into
account the local pressure drop. The whole system of nonlinear equations can be
introduced directly into the pressure’ network. This solution was first proposed by
Walton \cite{walton_1984} and Roldan \cite{roldan_1985} and has been used in various multizone models
developed since (Feustel \cite{feustel_1989}). It leads to the definition of various flow elements
in simulating one large opening.

A second way to solve this problem is to interpret the flow equations of large
openings in terms of non linear pressure laws. This method leads to the definition
of new flow equations in pressure characterizing the behavior of large openings.
Walton \cite{walton_1989} proposed this solution in the case of a vertical opening between two
isothermal zones i and j for steady state flow. Following the approximation of the
Bernoulli flow it is assumed that the velocity of the air flow at different heights is
given by the orifice equation (Eq. (\ref{height-orifice-flow})):
\begin{equation}\label{height-orifice-flow}
v_{ij}(z) = \left[2\frac{P_i(z)-P_j(z)}{\rho}\right]^{1/2}
\end{equation}
where $\rho$ represents the density of flowing air.

The hydrostatic equation is then used to relate the pressures in each room to the
air characteristics at various heights, and the velocity equation becomes: 
\begin{equation}\label{height-velocity-equation}
v_{ij}(z) = \left[2\left(\frac{\left(P_{i,0}-P_{j,0}\right)-gz\left(\rho_i-\rho_j\right)}{\rho}\right)\right]^{1/2}
\end{equation}

The height of the neutral plane is given by $P_i (z) = P_j(z)$ or by setting the interzonal velocity to zero.
In the case of a neutral plane being located in the opening,
and taking this point as origin for altitude, since $P_i(zn) =P_j(zn)$, the mass flow
through the doorway above and below the neutral level $zn$ are given by Eqs. (\ref{upper-integral})
and (\ref{lower-integral}):
\begin{equation}\label{upper-integral}
\dot{m}_{zn,H} = Cd \int\limits_{z=0}^{z=H-zn}\rho v_{ij}(z) W dz
\end{equation}
%
\begin{equation}\label{lower-integral}
\dot{m}_{0,zn} = Cd \int\limits_{z=-zn}^{z=0}\rho v_{ij}(z) W dz
\end{equation}
%
Walton \cite{walton_1984} reports that this model of doorways tends to be faster than the multiple opening approach.
However it complicates the assembly process of the network since one or two flows may exist through the same link
between two zones. 

\subsubparagraph{COMIS Contribution}
The main objective in proposing a solution for large openings is to fit it easily
into the network definition and to go as far as possible in the modeling of the
various phenomena influencing the behavior of large openings.
The main assumptions of our model are:
\begin{itemize}
\item Steady flow, inviscid and incompressible fluid
\item Linear density stratification on both sides of the opening
\item Turbulence effects represented by an equivalent pressure difference profile
\item Effects of reduction of the effective area of the aperture represented by a single
coefficient 
\end{itemize}
\subsubparagraph{Examples of Solutions}
\subsubsection{Conclusion}

\bibliographystyle{unsrt}
\bibliography{Fundamentals}
\end{document}